\chapter{Conclusões e Trabalhos futuros}

Este trabalho mostrou como usar
o ruído de Perlin para poder criar biomas com relevo distinto em terrenos 3D onde as fronteiras entre os biomas são contínuas e suavizadas.
Utilizando uma combinação de algoritmos e métodos, apresentamos detalhes da implementação de uma ferramenta geradora de um ambiente virtual onde os 
terrenos são divididos entre planícies, montanhas, vales, desertos e cânyons. 
Esses biomas tem altura e cor distintas, mas é possível utilizar a mesma técnica para suavizar
outras características como: temperatura, vegetação, frequência de vegetação e vemos isso como continuação deste trabalho. 

A vantagem de utilizar uma abordagem não assistida para construção de um ambiente virtual é economizar esforço e tempo do desenvolvedor de jogos,
permitindo que a geração de terrenos seja automatizada.  
Isso pode ser facilmente usado para desenvolver um jogo na hora de distribuir 
a raridade entre os biomas e a probabilidade de proximidade entre eles. Isso é visível em
algumas das imagens \ref{fig:comparandofreqdebiomasyeah}.Essa técnica pode ajudar a desenvolver um terreno massivo para jogos com
múltiplas características de terreno separadas em áreas quadradas.

%A vantagem de usar o ruído para decidir qual bioma pertence a cada região é que ele mantêm biomas de valores parecidos próximos. 


\section{Trabalhos futuros}

\begin{itemize}
    \item Aplicar ruído unidimensional, nas fronteiras, desta maneira as áreas de fronteira
    não serão mais retângulos, e irão assumir uma forma mais orgânica.
    \item Dividir as áreas de biomas usando diagrama de Voronoi sem perder a característica de fronteiras contínuas.
    \item Usar tesselação, criando quantidade de polígonos no terreno por demanda, 
    isso vai aumentar o desempenho de \textit{frames} por segundo da aplicação. 
    Comparar resultados de desempenhos obtidos.
    \item Usar interpolações não lineares nas fronteiras.
    
\end{itemize}