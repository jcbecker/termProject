\chapter{Conclusões e Trabalhos futuros}
Nos algoritmos apresentados na sessão de implementação, é mostrado uma técnica para 
criar um terreno com múltiplas características de relevo chamados de bioma neste trabalho
onde a fronteira entre esses biomas é contínua e suavizada, as características interpoladas
neste trabalho foram altura e cor, mas é possível usar a mesma técnica para suavizar
outras características como: temperatura, vegetação, frequência de vegetação.
Essa técnica pode ajudar bastante a desenvolver um terreno massivo para jogos com
múltiplas características de terreno separadas em áreas quadradas.

\section{Trabalhos futuros}

\begin{itemize}
    \item Aplicar ruído unidimensional, nas fronteiras, desta maneira a área de fronteira
    não seria retângulos de $b \cdot l$ como são neste trabalho, as retas de de tamanho
    $b$ deixariam de ser retas para assumir uma forma mais orgânica.
    \item Dividir áreas de biomas com um formato não necessariamente quadrados de $b$ por $b$, 
    e dividir as áreas usando \textit{sites} do diagrama de Voronoi.
    \item Usar tesselação, criando quantidade de polígonos no terreno por demanda, 
    isso vai aumentar o desempenho de \textit{frames} por segundo da aplicação, durante esse trabalho 
    desempenho de \textit{fps} não foi levado em consideração por não ser objetivo deste projeto.
    
\end{itemize}