\chapter{Objetivos}

\section{Objetivo Geral}
Este projeto tem como objetivo gerar mapas de altura proceduralmente com ruído 
de Perlin, de maneira não assistida, os mapas de altura devem representar o 
relevo de pelo menos dois biomas arbitrários com fronteiras contínuas.

%Criar um método procedural para múltiplos biomas ter fronteiras contínuas

\section{Objetivo Específico}

\begin{itemize}
    \item Selecionar maneira de separar regiões;
    \item Selecionar biomas, e as caracteristicas dos mesmos a ser representadas;
    \item Construir algoritmo para manipular ruído de Perlin e gerar caracteristicas
        selecionadas do bioma;
    \item Implementar fronteiras contínuas entre biomas;
    \item Comparar resultado com cenários de jogos;
    \item Comparar resultados com a natureza.
  \end{itemize}


%Me parece que justificativa está na problemática
%\section{Justificativa}



%acho que preciso fazer uma seção ou capitulo apenas para a metodologia
%\section{Metodologia}