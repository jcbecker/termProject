\chapter{Revisão Bibliográfica}
Neste capitulo serão abordados alguns conceitos e técnicas necessárias para
gerar o mapa de altura proceduralmente, além de apresentar alguns trabalhos
relacionados.

\section{Biomas}
Existem mais que um conceito de bioma, um bastante adotado considera bioma como
uma área do espaço geográfico, representada por um tipo uniforme de ambiente, o
mesmo pode ser classificado de acordo com o macroclima, fitofisionomia (formação),
solo e a altitude, os elementos que maus caracterizam os ambientes continentais, 
este é o conceito de \cite{coutinho2006conceito}, usando como base descrições
de \cite{walter1986vegetaccao}.

Mas o conceito de bioma neste trabalho vai ser outro...%posso fazer isso?

%\textit{"considera como bioma
%uma  área  do  espaço  geográfico,  com  dimensões  até
%superiores  a  um  milhão  de  quilômetros  quadrados,
%representada  por  um  tipo  uniforme  de  ambiente,
%identificado  e  classificado  de  acordo  com  o
%macroclima,  a  fitofisionomia  (formação),  o  solo  e  a
%altitude, os principais elementos que caracterizam os
%diversos  ambientes  continentais."}\cite{coutinho2006conceito} e \cite{walter1986vegetaccao}.

\section{Representação das Regiões}

\section{Ruído de Perlin}

\section{Mapas de Altura}

\section{Trabalhos Relacionados}

%Ainda não sei ao certo quais são os trabalhos relacionados que vou apresentar
\subsection{Gabrielle e Carli}

\subsection{Bevilaqua}