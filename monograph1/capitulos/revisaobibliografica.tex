\chapter{Revisão Bibliográfica}
Neste capitulo serão abordados alguns conceitos e técnicas necessárias para
gerar o mapa de altura proceduralmente, além de apresentar alguns trabalhos
relacionados.

\section{Biomas} %Preciso de mais fontes nesta seção, urgente
Existem mais que um conceito de bioma, um bastante adotado considera bioma como
uma área do espaço geográfico, representada por um tipo uniforme de ambiente, o
mesmo pode ser classificado de acordo com o macroclima, fitofisionomia (formação),
solo e a altitude, os elementos que maus caracterizam os ambientes continentais, 
este é o conceito de \cite{coutinho2006conceito}, usando como base descrições
de \cite{walter1986vegetaccao}.

Porém neste trabalho o conceito de bioma vai ser outro, o resultado deste
vai ser unicamente mapas de altura, então a única característica que é relevante
são as altitudes do solo e seus padrões, descartando dados como tipo da
vegetação e sua distribuição, umidade e entre outros. Então no restante 
do trabalho, quando for usado os termos "característica do bioma", o mesmo vai
se referir a alguma característica de altitude do bioma.

Para fins de restringir o escopo do projeto vamos agrupar os biomas por padrões
de relevo e não por fauna e flora como feito no conceito apresentado acima,
estes biomas serão: cordilheiras (cadeias de montanha), planícies.

\section{Representação das Regiões}
Neste trabalho serão implementadas três maneiras de representar a malha de
regiões, malha de quadrados, malha triangular e diagrama de Voronoi.

Um bioma vai pertencer a um conjunto de regiões, na imagem \ref{fig:squadStripBiomes}
que usa uma malha de quadrados, onde os segmentos de retas vermelhas são fronteiras entre
biomas.
\begin{figure}[H]
    \centering
    \includegraphics[width=0.7\textwidth]{figuras/squadStripBiomes.png}
    \caption{Malha de quadrados com divisão de biomas}
    \label{fig:squadStripBiomes}
\end{figure}
\subsection{Malhas de quadrado}
Este modelo já apresentado na figura \ref{fig:squadStripBiomes}, nele cada
quadrado é uma região, os vértices armazenados se encontram nos quatro cantos
do quadrado, então um vértice é comum a quatro quadrados, cada um deles
compartilhando o vértice com os quadrados adjacentes, as arestas entre vértices
vizinhos são uma fronteira entre regiões, e tem a possibilidade desta aresta ser
também uma fronteira entre biomas.

Devido ao padrão podemos perceber que não temos a necessidade de armazenar arestas
em memória, já que a mesma só vai existir entre vértices vizinhos, o vértice
$v_{i j}$ tem como vizinhos o conjunto $\{v_{i+1 j}, v_{i-1 j}, v_{i j+1}, v_{i j-1}\}$.
\subsection{Malha triangular}
Usando a mesma base de vértice da malha de quadrados, agora temos uma aresta
adicional, uma diagonal em cada quadrado do modelo anterior, dividindo a região
em duas, cada triangulo sendo uma região, como podemos ver na figura \ref{fig:vbo}.
Agora além do conjunto de arestas do modelo anterior, o vértice $v_{i j}$ também tem
aresta para os vértices $\{v_{i+1 j+1}, v_{i-1 j-1}\}$
\begin{figure}[H]
    \centering
    \includegraphics[width=0.5\textwidth]{figuras/vbo.png}
    \caption{Malhas de triângulos, retirado de \cite{androidtrianglestrip}}
    \label{fig:vbo}
\end{figure}


\subsection{Diagrama de Voronoi}
Um diagrama de Voronoi é uma partição no plano para separar regiões, recebendo
como entrada um conjunto de pontos chamados de \textit{sites}, o algoritmo
separa as regiões de cada \textit{site} deixando a fronteira equidistante entre eles
\cite{fortune1987sweepline}.

Como está implementação vai ser não assistida estes sites precisam ser colocados
aleatoriamente e proceduralmente, para não ocorrer aglomerações de site, já
que existe a possibilidade de acontecer na geração de \textit{sites} aleatórios,
será usado o algoritmo de Lloyd's para relaxar os sites, o conjunto dessas
técnicas já foi usado por \cite{patel2010polygonal}, para gerar malha de regiões,
segue uma imagem de seu diagrama na figura \ref{fig:voronoi-2-lloyd}, este diagrama
foi usado para alcançar o resultado já mostrado na ilustração \ref{fig:voronoi-map-goal-distorted}.
\begin{figure}[H]
    \centering
    \includegraphics[width=0.5\textwidth]{figuras/voronoi-2-lloyd.png}
    \caption{Diagrama de Voronoi com algoritmo de Lloyd's aplicado, por \cite{patel2010polygonal}}
    \label{fig:voronoi-2-lloyd}
\end{figure}

\section{Ruído de Perlin}
Bons geradores de números aleatórios geram números onde não existe relação entre
eles, porem se montado um gráfico com eles, o resultado não teria um aspecto
orgânico, então para o terreno ter um relevo mais parecido com o encontrado na 
natureza é usado uma função de ruído \cite{shiffman2012nature}, está comparação
pode ser feita com a imagem \ref{fig:randomAndNoise}. 
\begin{figure}[H]
    \centering
    \includegraphics[width=0.7\textwidth]{figuras/randomAndNoise.png}
    \caption{Da esquerda, função de ruído, função de pontos aleatórios. Por \cite{shiffman2012nature}}
    \label{fig:randomAndNoise}
\end{figure}

Uma função de ruído recebe como parâmetro as coordenadas em um espaço de $n$ dimensões,
retorna um valor entre $0$ e $1$ para tal posição\cite{shiffman2012nature}.

Na figura \ref{fig:randomAndNoise} vimos o ruído unidimensional, o ruído
tem a complexidade $O(2^n)$, sendo $n$ a quantidade de dimensões \cite{zucker2001perlin}.
Este mesmo ruído unidimensional pode ser usado nas fronteiras entre biomas, dando
um aspecto mais natural á fronteiras, da mesma maneira que foi usado nas bordas
das ilhas no trabalho de \cite{patel2010polygonal}. Contudo, para gerar altura
em um cenário tridimensional precisamos usar o ruído bidimensional, existem
implementações do ruído em placas gráficas que calculam a função de ruído para 
$\{1, 2, 3\}$ dimensões em único ciclo \cite{perlin2002improving}.

No ruído unidimensional a resposta é uma interpolação entre seus vizinhos, que
neste caso são apenas $2$, caso o parâmetro do ruído for $x_{i}$, o mesmo só tem 
como vizinhos, $x_{i+1}$ e $x_{i-1}$, quando a dimensão aumenta e por conta disso
os parâmetros também, a quantidade de vizinhos também aumenta \cite{shiffman2012nature}, 
como o mesmo exemplifica na ilustração \ref{fig:1dto2dnoise}.
\begin{figure}[H]
    \centering
    \includegraphics[width=0.7\textwidth]{figuras/1dto2dnoise.png}
    \caption{Vizinhas para função de ruído, $1d$ para $2d$. Por \cite{shiffman2012nature}}
    \label{fig:1dto2dnoise}
\end{figure}

O ruído de Perlin usa a função ruído explicada acima várias vezes, cada uma é chamada
de oitava, cada oitava usa amplitude e frequência diferente, a quantidade de oitavas
usada varia de implementações para necessidades, a primeira oitava é gerada com uma
amplitude alta e frequência baixa, as próxima oitava tem a metade da amplitude e 
o dobro da frequência, o ruído de Perlin é a soma de todas as oitavas computadas,
adaptado de Hugo Elias \cite{carli2012canion} gerou as imagens
\ref{fig:perlin1d} e \ref{fig:perlin2d} do ruído de Perlin em uma e duas
dimensões respectivamente.
\begin{figure}[H]
    \centering
    \includegraphics[width=0.7\textwidth]{figuras/perlin1d.png}
    \caption{Gerando ruído de Perlin para uma dimensão}
    \label{fig:perlin1d}
\end{figure}
\begin{figure}[H]
    \centering
    \includegraphics[width=0.6\textwidth]{figuras/perlin2d.png}
    \caption{Gerando ruído de Perlin para duas dimensões}
    \label{fig:perlin2d}
\end{figure}

\section{Mapas de Altura}
Mapas de altura é uma maneira de representar altitudes de uma plano, e essa
será uma das saídas desta implementação proposta, mapas de altura costuma ser uma imagem
onde os pontos mais claros representam pontos mais elevados e os escuros regiões mais baixas.
A imagem \ref{fig:perlin2d} já traz exemplos de mapas de altura.

\section{Trabalhos Relacionados}

%correção: preciso colocar trabalhos fora deste vínculo e quem sabe adicionar uma seção
%sobre
%Ainda não sei ao certo quais são os trabalhos relacionados que vou apresentar
\subsection{Trabalho de Carli}

