\chapter{Metodologia}
Nesta seção veremos a metodologia do trabalho desde a busca por fontes e trabalhos
relacionados até a analise de resultados.

Começando pela leitura de trabalhos relacionados, os primeiros trabalhos lidos
foram recomendados por professores, a partir destes trabalhos li algumas de suas
referências mais relevantes, os mesmos continham outras citações relevantes, e
assim por diante, dessa maneira, tenho em mãos uma "árvore" de citações,
para preencher algumas lacunas e reforçar alguns argumentos foram necessárias
mais buscas de artigos e trabalhos.

%Selecionar maneira de separar regiões;
A primeira etapa da implementação é escolher a maneira em que regiões serão
separadas, um bioma vai pertencer a um conjunto de regiões, as opções
pré-selecionadas são: \textit{Triangle strip}, malha de quadrados e diagrama de
voronoi. Cada uma delas tem sua vantagem, diagrama de voronoi constrói fronteiras
mais naturais, as outras técnicas tem implementação mais fácil, menor custo
computacional e geram a possibilidade de trabalhar com mapas pseudo infinitos.
Então serão implementadas as três e feitas comparações sobre elas para decidir
qual a melhor opção. A decisão será tomada...
NÃO FAÇO A MENOR IDEIA DE COMO ESSA DECISÃO VAI SER FEITA.

%Selecionar biomas, e as características dos mesmos a ser representadas;
Próxima etapa é a escolha de biomas e suas características a serem geradas pelo%estou repetindo de mais a palavra caracteristica, preciso dar um jeito nisso
algoritmo, a principio a escolha será feita por características fáceis de
implementar, biomas com características mais distintas, para ser visualmente
mais fácil de distinguir regiões de um e outro.%Construir algoritmo para manipular ruído de Perlin e gerar características selecionadas do bioma;
E como já visto no trabalho de \cite{carli2012canion}, podemos manipular o ruído
de Perlin para imitar as características de alguns padrões da natureza, então, 
o mesmo será feito para os biomas selecionados.

%Implementar fronteiras contínuas entre biomas;
Caso não tratado, as fronteiras em biomas podem e provavelmente serão
descontinuas, tendo uma mudança brusca de altura nas fronteiras e deixando o
cenário sem sentido, então nesta etapa será analisadas as possibilidades para
correção da descontinuidade e implementar a mesma.

%Comparar resultado com cenários de jogos e Comparar resultados com a natureza.