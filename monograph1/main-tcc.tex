%\listfiles
\documentclass[tg]{mdtuffs}
% um tipo específico de monografia pode ser informado como parâmetro opcional:
%\documentclass[tese]{mdtuffs}
% a opção `openright' pode ser usada para forçar inícios de capítulos
% em páginas ímpares
% \documentclass[openright]{mdtuffs}
% para gerar uma versão frente-e-verso, use a opção 'twoside':
% \documentclass[twoside]{mdtuffs}
%\usepackage{hyperref}	
%\usepackage{breakurl}
\usepackage[T1]{fontenc}        % pacote para conj. de caracteres correto
\usepackage{fix-cm} %para funcionar corretamente o tamanho das fontes da capa
\usepackage{times, color,xcolor}       % pacote para usar fonte Adobe Times e cores
\usepackage[utf8]{inputenc}   % pacote para acentuação
\usepackage{graphicx}  % pacote para importar figuras
\usepackage{caption}

%\usepackage[brazil]{babel}   
\usepackage{enumerate}
\usepackage{amsmath,latexsym,amssymb} %Pacotes matemáticos
\usepackage[%hidelinks%, 
            bookmarksopen=true,linktoc=none,colorlinks=true,
            linkcolor=black,citecolor=black,filecolor=magenta,urlcolor=blue,
            pdftitle={Título da Dissertação ou Trabalho ....},
            pdfauthor={Nome Autor Sobrenome},
            pdfsubject={Projeto/Trabalho de Conclusão de Curso},
            pdfkeywords={Monografia, Modelo, LaTeX}
            ]{hyperref} %hidelinks disponível no pacote hyperref a partir da versão 2011-02-05  6.82a
%Nesse caso, hidelinks retira os retângulos em volta dos links das referências

%Definição de minha autoria pra funcionar as XML
%\def\lnl#1#2{{\scriptsize\bfseries #1}\hspace{#2em}}
\usepackage{setspace}
%---------------------------------------------------
% Definições dos XML
%---------------------------------------------------
\usepackage{verbatim}
\usepackage{listings}
%\usepackage[usenames,dvipsnames]{color}
	%\def\mlnl#1#2{{\scriptsize\bfseries #1}\hspace{#2em}}

%---------------------------------------------------
% Definições dos Algoritmos
%---------------------------------------------------
	\usepackage[portuguese,ruled,linesnumbered]{algorithm2e}
	\usepackage{etoolbox}
	\makeatletter
	\patchcmd{\@algocf@start}{%
		\begin{lrbox}{\algocf@algobox}%
	}{%
	  \rule{0.\textwidth}{\z@}%
	  \begin{lrbox}{\algocf@algobox}%
	  \begin{minipage}{1.\textwidth}%
	}{}{}
	\patchcmd{\@algocf@finish}{%
	  \end{lrbox}%
	}{%
	  \end{minipage}%-----------------------------------------------------------Tem erro de sintaxe aqui
	  \end{lrbox}%
	}{}{}
	\makeatother

	\SetAlFnt{\tt}
	
	\SetKwFunction{FRecurs}{FnRecursive}%
%-------------------------------------------------

     
\sloppy
%%Margens conforme MDT 1ª edição, arrumar diretamente no mdtuffs.cls para funcionar a opção twoside *PENDENTE*
\usepackage[inner=30mm,outer=20mm,top=30mm,bottom=20mm]{geometry} 

%==============================================================================
% Se o pacote hyperref foi carregado a linha abaixo corrige um bug na hora
% de montar o sumário da lista de figuras e tabelas
% Se o pacote não foi carregado, comentar a linha %
%==============================================================================

%%=============================================================================
%% Trampa para corrigir o bug do hyperref que redefine o caption das figuras e das
%% tabelas, n�o colocando o nome ``Figura'' antes do n�mero do mesmo na lista
%%=============================================================================

\makeatletter

\long\def\@caption#1[#2]#3{%
  \expandafter\ifx\csname if@capstart\expandafter\endcsname
                  \csname iftrue\endcsname
    \global\let\@currentHref\hc@currentHref
  \else
    \hyper@makecurrent{\@captype}%
  \fi
  \@ifundefined{NR@gettitle}{%
    \def\@currentlabelname{#2}%
  }{%
    \NR@gettitle{#2}%
  }%
  \par\addcontentsline{\csname ext@#1\endcsname}{#1}{%
    \protect\numberline{\csname fnum@#1\endcsname ~-- }{\ignorespaces #2}%
  }%
  \begingroup
    \@parboxrestore
    \if@minipage
      \@setminipage
    \fi
    \normalsize
    \expandafter\ifx\csname if@capstart\expandafter\endcsname
                    \csname iftrue\endcsname
      \global\@capstartfalse
      \@makecaption{\csname fnum@#1\endcsname}{\ignorespaces#3}%
    \else
      \@makecaption{\csname fnum@#1\endcsname}{%
        \ignorespaces
        \ifHy@nesting
          \expandafter\hyper@@anchor\expandafter{\@currentHref}{#3}%
        \else
          \Hy@raisedlink{%
            \expandafter\hyper@@anchor\expandafter{%
              \@currentHref
            }{\relax}%
          }%
          #3%
        \fi
      }%
    \fi
    \par
  \endgroup
}

\makeatother

%==============================================================================
% Identificação do trabalho
%==============================================================================
\title{Geração procedural de biomas}

\author{Becker}{João Carlos}
%Descomentar se for uma "autora"
%\autoratrue

\course{Curso de Ciência da Computação}
\altcourse{Curso de Ciência da Computação}

%não usado por enquanto
\institute{Ciência da Computação}
\degree{Bacharel em Ciência da Computação}

% Número do TG (verificar na secretaria do curso)
% Para mestrado deixar sem opção dentro do {}
\trabalhoNumero{}

%Orientador
\advisor[Prof.]{Dr.}{Wuerges}{Emílio}
%Se for uma ``orientadora'' descomentar a linha baixo
%\orientadoratrue

%Co orientador, comentar se não existir
%\coadvisor[Prof.]{Drª.}{Pereira}{Maria Regina}
%\coorientadoratrue %Se for uma ``Co-Orientadora''

%Avaliadores (Banca)
\committee[Dr.]{Sobrenome}{Nome}{UFFS}
\committee[Me.]{Sobrenome}{Nome}{UFFS}

% a data deve ser a da defesa; se nao especificada, são gerados
% mes e ano correntes
\date{30}{Junho}{2017}

%Palavras chave
%\keyword{Dissertação} 
%\keyword{Modelo}
%\keyword{LaTeX}

%%=============================================================================
%% Início do documento
%%=============================================================================
\begin{document}

%%=============================================================================
%% Capa e folha de rosto
%%=============================================================================
\maketitle

%%=============================================================================
%% Catalogação e Folha de aprovação
%%=============================================================================
\clearpage
Somente para TCC 2: Está página deve ser substituida pela ficha de catalogação antes de sua entrega na biblioteca. 
%Somente obrigatório para dissertação, para TG, remover as linhas	77	%
%Como a CIP vai ser impressa atrás da página de rosto, as margens inner e outer	
%devem ser invertidas.
%\newgeometry{inner=20mm,outer=30mm,top=30mm,bottom=20mm}	
%\makeCIP{email@email.com} %email do autor		
%\restoregeometry

%Se for usar a catalogação gerada pelo gerador do site da biblioteca comentar as linhas
%acima e utilizar o comando abaixo
%\includeCIP{CIP.pdf}

%folha de aprovação

%IMPORTANTE: Esta folha deverá ser impressa e assinada pela banca e depois digitalizada e inserida no arquivo final para entrega no repositório institucional.
\makeapprove

%%=============================================================================
%% Dedicatória (opcional)
%%=============================================================================
%\clearpage
%\begin{flushright}
%\begin{onehalfspacing}
%\mbox{}\vfill

%{Texto da dedicatória ......}

%\end{onehalfspacing}
%\end{flushright}

%%=============================================================================
%% Agradecimentos (opcional)
%%=============================================================================
%\clearpage
%{
%\centering
%\textbf{AGRADECIMENTOS}\\
%}
%\vspace{1.5cm}
%\begin{onehalfspacing}

%Texto de agradecimento. 

%\end{onehalfspacing}


%%=============================================================================
%% Epígrafe (opcional)
%%=============================================================================
%\clearpage
%\begin{flushright}
%\mbox{}\vfill
%{\sffamily\itshape
%``Frase da epígrafe'' \\ }
%--- \textsc{Autor da frase}
%\end{flushright}


%%=============================================================================
%% Resumo
%%=============================================================================
\begin{abstract}
Resumo...
\end{abstract}

%%=============================================================================
%% Abstract
%%=============================================================================
% resumo na outra língua
% como parametros devem ser passados o titulo, o nome do curso,
% as palavras-chave na outra língua, separadas por vírgulas, o mês em inglês
%o a sigla do dia em inglês: st, nd, th ...
\begin{englishabstract}
{Title}
{Bachelor of Computer Science}
{Keywords1. Keyword2}
{March}
{st}
Abstract... 
\end{englishabstract}

%% Lista de Ilustrações (opc)
%% Lista de Símbolos (opc)
%% Lista de Anexos e Apêndices (opc)

%%=============================================================================
%% Lista de figuras (comentar se não houver)
%%=============================================================================
\listoffigures

%%=============================================================================
%% Lista de tabelas (comentar se não houver)
%%=============================================================================
\listoftables

%%=============================================================================
%% Lista de Apêndices (comentar se não houver)
%%=============================================================================
%\listofappendix

%%=============================================================================
%% Lista de Anexos (comentar se não houver)
%%=============================================================================
%\listofannex

%%=============================================================================
%% Lista de abreviaturas e siglas
%%=============================================================================
 %o parametro deve ser a abreviatura mais longa
%\begin{listofabbrv}{UbiComp}
%   \item [BNF] \textit{Backus-Naur Form}
 %  \item [UbiComp] Computação Ubíqua
%\end{listofabbrv}


%%=============================================================================
%% Lista de simbolos (opcional)
%%=============================================================================
%Simbolos devem aparecer conforme a ordem em que aparecem no texto
% o parametro deve ser o símbolo mais longo
%\begin{listofsymbols}{teste}
 % \item [$\varnothing$] vazio
 % \item [$\Gamma$]  Gama
 % \item [$\forall$] Para todo
%\end{listofsymbols}

%%=============================================================================
%% Sumário
%%=============================================================================
\tableofcontents


%%=============================================================================
%% Início da monografia
%%=============================================================================
\setlength{\baselineskip}{1.5\baselineskip}

%Adiciona cada capitulo
% comando para correção ortográfica
%$ aspell -t -c introducao.tex --encoding=utf-8 --lang=pt_BR

\chapter{Introdução}

\section{Problemática}
O consumo de jogos é crescente nos últimos 5 anos, como podemos ver na figura
\ref{fig:esa_graph_2017} das pesquisas da ESA,
segundo os mesmos, em 2016 o consumidor norte americano gastou em torno de $30.4$
bilhões de dólares na industria dos jogos. No mesmo ano, as companhias de jogos
norte americanas adicionaram mais de 11.7 bilhões de dólares na \textit{GDP}
do país \cite{entertainment2017essential}.
\begin{figure}[H]
%    \captionsetup{justification=raggedright, singlelinecheck=false}
    \centering
    \includegraphics[width=0.5\textwidth]{figuras/ESAGraph2017.png}
    \caption{Consumo com conteúdo de jogos\cite{entertainment2017essential}}
    \label{fig:esa_graph_2017}
\end{figure}

Em contrapartida, os investimentos nos jogos também são crescentes neste
período \cite{entertainment2017essential}, um dos jogos com mais investimentos é
o \textit{GTAV} que em desenvolvimento e marketing gastou cerca de $265$
milhões de dólares \cite{villapaz2013gta}. Os jogos tendem a ser cada vez mais,
detalhados e complexos, desta maneira o custo de desenvolvimento também aumenta.
No desenvolvimento de jogos temos vários profissionais envolvidos para criar
o conteúdo dos jogos, com equipes de programadores, designers,  roteiristas,
entre outros. A força de trabalho dos mesmos costuma ser a parte mais cara da
criação do jogo.

\section{Apresentação}
Uma maneira de conseguir diminuir os gastos no desenvolvimento de conteúdo, é 
gerando os mesmos proceduralmente, as chamadas técnicas de \textit{PCG}. 
\textit{PCG} é usar algoritmos para gerar o conteúdo \cite{shaker2016procedural}.
Uma aplicação bem comum do \textit{PCG} é a criação de relevos e mapas de altura,
desta maneira não é necessário uma pessoa modelar manualmente a altura do
terreno do cenário.

Com o \textit{PCG} para criar o terreno, é possível criar eles com tamanhos
pseudo-infinitos, hoje temos diversos exemplos de jogos que fazem uso dessa técnica
para criar um cenário pseudo-infinitos, entre eles, \textit{Limit Theory}, nele
são criados pseudo-infinitos sistemas planetários, de forma procedural, e em cada
sistema planetário os planetas e seus relevos também são gerados proceduralmente
\cite{abreu1990toward}.

O algoritmo para criar o terreno deve ser implementado conforme as
características do bioma alvo, como exemplo, a implementação de 
\cite{gabrielle2016canion} e \cite{carli2012canion}, ambos geram relevos de
cânions, como podemos visualizar na figura \ref{fig:carli2012result}.
\begin{figure}[H]
%    \captionsetup{justification=raggedright, singlelinecheck=false}
    \centering
    \includegraphics[width=0.9\textwidth]{figuras/carli2012result.png}
    \caption{Resultado do trabalho de \cite{carli2012canion}}
    \label{fig:carli2012result}
\end{figure}

Na técnica de \cite{patel2010polygonal}, que gera ilha proceduralmente, cada uma podendo ter
múltiplos biomas. Para começar é criado um diagrama de voronoi, onde cada \textit{site}
vai representar uma região do mapa, os \textit{sites} são gerados em posições
aleatórias, seguido por Lloyd's algorithm no diagrama para deixar as regiões mais relaxadas \ref{fig:voronoi-2-lloyd}.
Então é usados uma sequencia de ruídos bidimensionais sobre o mapa para decidir
áreas de oceano ou ilha, altura da região e bioma da região \ref{fig:patel2010tablebiomes} e \ref{fig:patel2010biomes}, seguido por ruídos
unidimensionais para as bordas da ilha e fronteiras das regiões ter uma aparência mais natural, 
por fim tendo como resultado final ilustrado na figura. 
\begin{figure}[H]
%    \captionsetup{justification=raggedright, singlelinecheck=false}
    \centering
    \includegraphics[width=0.6\textwidth]{figuras/voronoi-2-lloyd.png}
    \caption{Diagrama de voronoi com Lloyd's aplicado}
    \label{fig:voronoi-2-lloyd}
\end{figure}

\begin{figure}[H]
%    \captionsetup{justification=raggedright, singlelinecheck=false}
    \centering
    \includegraphics[width=0.4\textwidth]{figuras/patel2010tablebiomes.png}
    \caption{Tabela de biomas}
    \label{fig:patel2010tablebiomes}
\end{figure}

\begin{figure}[H]
%    \captionsetup{justification=raggedright, singlelinecheck=false}
    \centering
    \includegraphics[width=0.4\textwidth]{figuras/patel2010biomes.png}
    \caption{Regiões com bioma}
    \label{fig:patel2010biomes}
\end{figure}

\begin{figure}[H]
%    \captionsetup{justification=raggedright, singlelinecheck=false}
    \centering
    \includegraphics[width=0.4\textwidth]{figuras/voronoi-map-goal-distorted.png}
    \caption{Resultado final de \cite{patel2010polygonal}}
    \label{fig:voronoi-map-goal-distorted}
\end{figure}

O jogo \textit{minecraft} tem uma implementação de geração procedural, seus mundos são de tamanho pseudo-infinito, 
o algoritmo é não assistido, ou seja, sem a intervenção do usuário na criação do mundo, nele
são gerados múltiplos biomas, em cada um deles existe uma manipulação diferente
do ruído para recriar as características do bioma. Mesmo em seu mundo
minimalista, o jogador consegue reconhecer o bioma \cite{short2012teaching}, como na figura \ref{fig:biomesminecraftgameplay}.
Foi usado o sistema online \textit{Chunck Base} para 
ilustrar fronteiras entre biomas, cada cor no mapa é um bioma \ref{fig:chunkbasebiomes}.


\begin{figure}[H]
%    \captionsetup{justification=raggedright, singlelinecheck=false}
    \centering
    \includegraphics[width=0.7\textwidth]{figuras/chunkbasebiomes.png}
    \caption{Mapa de biomas para mundo virtual do \textit{minecraft}}
    \label{fig:chunkbasebiomes}
\end{figure}

\begin{figure}[H]
%    \captionsetup{justification=raggedright, singlelinecheck=false}
    \centering
    \includegraphics[width=0.7\textwidth]{figuras/biomesminecraftgameplay.png}
    \caption{Perspectiva de um jogador no \textit{minecraft}}
    \label{fig:biomesminecraftgameplay}
\end{figure}
%\subsubsection{Exemplo de ilustração}

%\begin{figure}[h]
%\captionsetup{justification=raggedright, singlelinecheck=false}
%\centering
%\includegraphics[scale=0.4]{figuras/uffs.eps}
%\caption{Exemplo de ilustração.}
%\label{fig:uffs}
%\end{figure}



%Exemplo do Matheus
%\begin{figure}[H]
%  \centering
%  \includegraphics[width=0.9\textwidth]{figuras/tcc2/pts_sem_peso/evento_de_site_insercao_fronteiras.png}
%  \caption{ES para o \textit{site} $s$.}
%  \label{fig:cenario_insercao_de_fronteiras}
%\end{figure}
%
%  Levando em consideração o cenário da figura \ref{fig:cenario_insercao_de_fronteiras}, 
%$s$ está contido em $R^{*}_{q}$, portanto serão criadas as fronteiras $C^{-}_{qs}$ e $C^{+}_{qs}$,

\chapter{Trabalhos Relacionados}

Exemplo: ~\cite{denio2008}.


\setlength{\baselineskip}{\baselineskip}

%%=============================================================================
%% Referências
%%=============================================================================
%\bibliographystyle{abbrv}------------------------------------------------------Original usava essa, mas sem informações da fonte, troca para abnt
%\bibliography{referencias/referencias}
\bibliographystyle{abnt}
\bibliography{referencias/referencias}



%IMPORTANTE: Se precisar usar alguma seção ou subseção dentro dos apêndices ou
%anexos, utilizar o comando \tocless para não adicionar no Sumário
%Exemplos: 
% \tocless\section{Histórico}
%%=============================================================================
%% Apêndices
%%=============================================================================
%\appendix
%\include{capitulos/apendicea}
%\include{capitulos/apendiceb}

%%=============================================================================
%% Anexos
%%=============================================================================
%\annex
%\include{capitulos/anexoa}

\end{document}
