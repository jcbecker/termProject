%\listfiles
\documentclass[tg]{mdtuffs}
% um tipo específico de monografia pode ser informado como parâmetro opcional:
%\documentclass[tese]{mdtuffs}
% a opção `openright' pode ser usada para forçar inícios de capítulos
% em páginas ímpares
% \documentclass[openright]{mdtuffs}
% para gerar uma versão frente-e-verso, use a opção 'twoside':
% \documentclass[twoside]{mdtuffs}
%\usepackage{hyperref}	
%\usepackage{breakurl}
\usepackage[T1]{fontenc}        % pacote para conj. de caracteres correto
\usepackage{fix-cm} %para funcionar corretamente o tamanho das fontes da capa
\usepackage{times, color,xcolor}       % pacote para usar fonte Adobe Times e cores
\usepackage[utf8]{inputenc}   % pacote para acentuação
\usepackage{graphicx}  % pacote para importar figuras
\usepackage{caption}

%\usepackage[brazil]{babel}   
\usepackage{enumerate}
\usepackage{amsmath,latexsym,amssymb} %Pacotes matemáticos
\usepackage[%hidelinks%, 
            bookmarksopen=true,linktoc=none,colorlinks=true,
            linkcolor=black,citecolor=black,filecolor=magenta,urlcolor=blue,
            pdftitle={Título da Dissertação ou Trabalho ....},
            pdfauthor={Nome Autor Sobrenome},
            pdfsubject={Projeto/Trabalho de Conclusão de Curso},
            pdfkeywords={Monografia, Modelo, LaTeX}
            ]{hyperref} %hidelinks disponível no pacote hyperref a partir da versão 2011-02-05  6.82a
%Nesse caso, hidelinks retira os retângulos em volta dos links das referências

%Definição de minha autoria pra funcionar as XML
%\def\lnl#1#2{{\scriptsize\bfseries #1}\hspace{#2em}}
\usepackage{setspace}
%---------------------------------------------------
% Definições dos XML
%---------------------------------------------------
\usepackage{verbatim}
\usepackage{listings}
%\usepackage[usenames,dvipsnames]{color}
	%\def\mlnl#1#2{{\scriptsize\bfseries #1}\hspace{#2em}}

%---------------------------------------------------
% Definições dos Algoritmos
%---------------------------------------------------
	\usepackage[portuguese,ruled,linesnumbered]{algorithm2e}
	\usepackage{etoolbox}
	\makeatletter
	\patchcmd{\@algocf@start}{%
		\begin{lrbox}{\algocf@algobox}%
	}{%
	  \rule{0.\textwidth}{\z@}%
	  \begin{lrbox}{\algocf@algobox}%
	  \begin{minipage}{1.\textwidth}%
	}{}{}
	\patchcmd{\@algocf@finish}{%
	  \end{lrbox}%
	}{%
	  \end{minipage}%-----------------------------------------------------------Tem erro de sintaxe aqui
	  \end{lrbox}%
	}{}{}
	\makeatother

	\SetAlFnt{\tt}
	
	\SetKwFunction{FRecurs}{FnRecursive}%
%-------------------------------------------------

     
\sloppy
%%Margens conforme MDT 1ª edição, arrumar diretamente no mdtuffs.cls para funcionar a opção twoside *PENDENTE*
\usepackage[inner=30mm,outer=20mm,top=30mm,bottom=20mm]{geometry} 

%==============================================================================
% Se o pacote hyperref foi carregado a linha abaixo corrige um bug na hora
% de montar o sumário da lista de figuras e tabelas
% Se o pacote não foi carregado, comentar a linha %
%==============================================================================

%%=============================================================================
%% Trampa para corrigir o bug do hyperref que redefine o caption das figuras e das
%% tabelas, n�o colocando o nome ``Figura'' antes do n�mero do mesmo na lista
%%=============================================================================

\makeatletter

\long\def\@caption#1[#2]#3{%
  \expandafter\ifx\csname if@capstart\expandafter\endcsname
                  \csname iftrue\endcsname
    \global\let\@currentHref\hc@currentHref
  \else
    \hyper@makecurrent{\@captype}%
  \fi
  \@ifundefined{NR@gettitle}{%
    \def\@currentlabelname{#2}%
  }{%
    \NR@gettitle{#2}%
  }%
  \par\addcontentsline{\csname ext@#1\endcsname}{#1}{%
    \protect\numberline{\csname fnum@#1\endcsname ~-- }{\ignorespaces #2}%
  }%
  \begingroup
    \@parboxrestore
    \if@minipage
      \@setminipage
    \fi
    \normalsize
    \expandafter\ifx\csname if@capstart\expandafter\endcsname
                    \csname iftrue\endcsname
      \global\@capstartfalse
      \@makecaption{\csname fnum@#1\endcsname}{\ignorespaces#3}%
    \else
      \@makecaption{\csname fnum@#1\endcsname}{%
        \ignorespaces
        \ifHy@nesting
          \expandafter\hyper@@anchor\expandafter{\@currentHref}{#3}%
        \else
          \Hy@raisedlink{%
            \expandafter\hyper@@anchor\expandafter{%
              \@currentHref
            }{\relax}%
          }%
          #3%
        \fi
      }%
    \fi
    \par
  \endgroup
}

\makeatother

%==============================================================================
% Identificação do trabalho
%==============================================================================
\title{Geração procedural de biomas}

\author{Becker}{João Carlos}
%Descomentar se for uma "autora"
%\autoratrue

\course{Curso de Ciência da Computação}
\altcourse{Curso de Ciência da Computação}

%não usado por enquanto
\institute{Ciência da Computação}
\degree{Bacharel em Ciência da Computação}

% Número do TG (verificar na secretaria do curso)
% Para mestrado deixar sem opção dentro do {}
\trabalhoNumero{}

%Orientador
\advisor[Prof.]{Dr.}{Wuerges}{Emílio}
%Se for uma ``orientadora'' descomentar a linha baixo
%\orientadoratrue

%Co orientador, comentar se não existir
%\coadvisor[Prof.]{Drª.}{Pereira}{Maria Regina}
%\coorientadoratrue %Se for uma ``Co-Orientadora''

%Avaliadores (Banca)
\committee[Dr.]{Sobrenome}{Nome}{UFFS}
\committee[Me.]{Sobrenome}{Nome}{UFFS}

% a data deve ser a da defesa; se nao especificada, são gerados
% mes e ano correntes
\date{30}{Junho}{2017}

%Palavras chave
%\keyword{Dissertação} 
%\keyword{Modelo}
%\keyword{LaTeX}

%%=============================================================================
%% Início do documento
%%=============================================================================
\begin{document}

%%=============================================================================
%% Capa e folha de rosto
%%=============================================================================
\maketitle

%%=============================================================================
%% Catalogação e Folha de aprovação
%%=============================================================================
\clearpage
Somente para TCC 2: Está página deve ser substituida pela ficha de catalogação antes de sua entrega na biblioteca. 
%Somente obrigatório para dissertação, para TG, remover as linhas	77	%
%Como a CIP vai ser impressa atrás da página de rosto, as margens inner e outer	
%devem ser invertidas.
%\newgeometry{inner=20mm,outer=30mm,top=30mm,bottom=20mm}	
%\makeCIP{email@email.com} %email do autor		
%\restoregeometry

%Se for usar a catalogação gerada pelo gerador do site da biblioteca comentar as linhas
%acima e utilizar o comando abaixo
%\includeCIP{CIP.pdf}

%folha de aprovação

%IMPORTANTE: Esta folha deverá ser impressa e assinada pela banca e depois digitalizada e inserida no arquivo final para entrega no repositório institucional.
\makeapprove

%%=============================================================================
%% Dedicatória (opcional)
%%=============================================================================
%\clearpage
%\begin{flushright}
%\begin{onehalfspacing}
%\mbox{}\vfill

%{Texto da dedicatória ......}

%\end{onehalfspacing}
%\end{flushright}

%%=============================================================================
%% Agradecimentos (opcional)
%%=============================================================================
%\clearpage
%{
%\centering
%\textbf{AGRADECIMENTOS}\\
%}
%\vspace{1.5cm}
%\begin{onehalfspacing}

%Texto de agradecimento. 

%\end{onehalfspacing}


%%=============================================================================
%% Epígrafe (opcional)
%%=============================================================================
%\clearpage
%\begin{flushright}
%\mbox{}\vfill
%{\sffamily\itshape
%``Frase da epígrafe'' \\ }
%--- \textsc{Autor da frase}
%\end{flushright}


%%=============================================================================
%% Resumo
%%=============================================================================
\begin{abstract}
Resumo...
\end{abstract}

%%=============================================================================
%% Abstract
%%=============================================================================
% resumo na outra língua
% como parametros devem ser passados o titulo, o nome do curso,
% as palavras-chave na outra língua, separadas por vírgulas, o mês em inglês
%o a sigla do dia em inglês: st, nd, th ...
\begin{englishabstract}
{Title}
{Bachelor of Computer Science}
{Keywords1. Keyword2}
{March}
{st}
Abstract... 
\end{englishabstract}

%% Lista de Ilustrações (opc)
%% Lista de Símbolos (opc)
%% Lista de Anexos e Apêndices (opc)

%%=============================================================================
%% Lista de figuras (comentar se não houver)
%%=============================================================================
\listoffigures

%%=============================================================================
%% Lista de tabelas (comentar se não houver)
%%=============================================================================
\listoftables

%%=============================================================================
%% Lista de Apêndices (comentar se não houver)
%%=============================================================================
%\listofappendix

%%=============================================================================
%% Lista de Anexos (comentar se não houver)
%%=============================================================================
%\listofannex

%%=============================================================================
%% Lista de abreviaturas e siglas
%%=============================================================================
 %o parametro deve ser a abreviatura mais longa
%\begin{listofabbrv}{UbiComp}
%   \item [BNF] \textit{Backus-Naur Form}
 %  \item [UbiComp] Computação Ubíqua
%\end{listofabbrv}


%%=============================================================================
%% Lista de simbolos (opcional)
%%=============================================================================
%Simbolos devem aparecer conforme a ordem em que aparecem no texto
% o parametro deve ser o símbolo mais longo
%\begin{listofsymbols}{teste}
 % \item [$\varnothing$] vazio
 % \item [$\Gamma$]  Gama
 % \item [$\forall$] Para todo
%\end{listofsymbols}

%%=============================================================================
%% Sumário
%%=============================================================================
\tableofcontents


%%=============================================================================
%% Início da monografia
%%=============================================================================
\setlength{\baselineskip}{1.5\baselineskip}

%Adiciona cada capitulo
\begin{frame}{Problemática}
    \begin{itemize} \setlength\itemsep{1em}
        \item Os jogos digitais estão cada vez melhores e exigindo mais
        complexidade para os mesmos, trazendo mais conteúdo agregado.
        \item O tempo para produzir este conteúdo demanda muito esforço de trabalho.
    \end{itemize}
\end{frame}


%\begin{frame}{Introdução}
%    
%\end{frame}

\begin{frame}{Objetivos}
    \begin{itemize}
        \item Objetivo Geral
        \begin{itemize}
            \item Este projeto tem como objetivo gerar mapas de tamanho pseudo-infinitos, com 
            relevo gerado proceduralmente usando ruído 
            de Perlin, de maneira não assistida, os mapas de altura devem representar o 
            relevo de pelo menos dois biomas arbitrários com fronteiras contínuas.
        \end{itemize}
        \item Objetivos Específicos
        \begin{itemize}
            \item Implementar malhas da superfície com tamanho pseudo-infinito;
            \item Selecionar biomas, e as características dos mesmos a ser representadas;
            \item Construir algoritmo para manipular ruído de Perlin e gerar características
                selecionadas do bioma;
            \item Gerar divisões entre biomas sobre a malha de regiões;
            \item Implementar fronteiras contínuas entre biomas;
            \item Comparar resultado com cenários de jogos.
        \end{itemize}
    \end{itemize}
\end{frame}
\chapter{Projeto}

\section{Objetivo Geral}

\section{Objetivo específico}

\section{Justificativa}

\section{Metodologia}
\chapter{Trabalhos Relacionados}

Exemplo: ~\cite{denio2008}.


\setlength{\baselineskip}{\baselineskip}

%%=============================================================================
%% Referências
%%=============================================================================
%\bibliographystyle{abbrv}------------------------------------------------------Original usava essa, mas sem informações da fonte, troca para abnt
%\bibliography{referencias/referencias}
\bibliographystyle{abnt}
\bibliography{referencias/referencias}



%IMPORTANTE: Se precisar usar alguma seção ou subseção dentro dos apêndices ou
%anexos, utilizar o comando \tocless para não adicionar no Sumário
%Exemplos: 
% \tocless\section{Histórico}
%%=============================================================================
%% Apêndices
%%=============================================================================
%\appendix
%\include{capitulos/apendicea}
%\include{capitulos/apendiceb}

%%=============================================================================
%% Anexos
%%=============================================================================
%\annex
%\include{capitulos/anexoa}

\end{document}
