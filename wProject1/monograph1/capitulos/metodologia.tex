\chapter{Metodologia}
Nesta seção veremos a metodologia do trabalho desde a busca por fontes e trabalhos
relacionados até a analise de resultados.

Inicialmente foi feito a leitura de trabalhos relacionados, os primeiros
foram recomendados por professores. A partir deles li algumas de suas
referências mais relevantes, os mesmos continham outras citações relevantes,
desta maneira, tendo uma "árvore" de citações.
Para preencher algumas lacunas e reforçar alguns argumentos foram necessárias
mais buscas de artigos e trabalhos.

%correção tem muita informação não fundamentada neste paragrafo
%Selecionar maneira de separar regiões;
A primeira etapa da implementação é a malha de regiões, um
bioma vai pertencer a um conjunto de regiões, as opções
selecionadas são: \textit{Triangle strip}, malha de quadrados e diagrama de
Voronoi.

%correção: isso já vai ser feito agora
%Selecionar biomas, e as características dos mesmos a ser representadas;
Em seguida será realizada a geração do mapa de altura para cordilheiras e planícies.
Por possuir características de relevo mais distintas, facilita visualmente distinguir,
as regiões e suas fronteiras. %Construir algoritmo para manipular ruído de Perlin e gerar características selecionadas do bioma;
Como já visto no trabalho de \cite{carli2012canion}, podemos manipular o ruído
de Perlin para imitar as características de alguns padrões da natureza, então, 
o mesmo será feito para os biomas selecionados.

%Implementar fronteiras contínuas entre biomas;
Caso não tratado, as fronteiras em biomas podem e provavelmente serão
descontinuas, tendo uma mudança brusca de altura nas fronteiras e deixando o
cenário sem sentido e sem naturalidade, então nesta etapa será analisadas as
possibilidades para correção da descontinuidade e implementar a mesma.

%Comparar resultado com cenários de jogos e Comparar resultados com a natureza.
O resultado final será um mapa de altura para múltiplos biomas, para verificar
a qualidade do resultado será feita comparações com cenários de jogos, e uma
analise de possibilidades de agentes percorrer o cenário. Para finalizar fazer
um julgamento visual do resultado com os trabalhos já feitos  e cenários apresentados em jogos.