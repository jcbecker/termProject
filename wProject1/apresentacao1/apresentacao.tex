% \documentclass[handout,xcolor=pdftex,dvipsnames,table]{beamer}

%\documentclass[aspectratio=169]{beamer} % WIDESCREEN
\documentclass[t]{beamer} % Aspecto 4:3

\usepackage[utf8]{inputenc}
\usepackage[T1]{fontenc}
\usepackage[english,brazil]{babel}
%\usepackage{natbib}

\usepackage{tikz}

% usando tema personalizado.
% arquivo beamerthemeUFFS.sty deve estar no mesmo diretório do .tex
\usepackage{beamerthemeUFFS}
\usepackage{ragged2e}
\usepackage{setspace}
\usepackage{animate}
\usepackage{graphicx}
\usepackage{subcaption}


\hypersetup{
    pdfstartview={Fit},
    pdftitle={Geração procedural de biomas},
 	pdfsubject={ApresentacaoDefesaTCC1},
 	pdfauthor={João Carlos Becker}
}

%%%%%%%%%%%%%%%%%%%%%%%%%%%%%%%%%%%%%%%%%%%%
 
\title{Geração procedural de biomas}
\author{João Carlos Becker} 

%\date{28 de junho de 2016}
\institute{Ciência da Computação\\
Universidade Federal da Fronteira Sul - Chapecó}

%%%%%%%%%%%%%%%%%%%%%%%%%%%%%%%%%%%%%%%%%%%%

\begin{document}

\setstretch{1} %espaçamento

\addtobeamertemplate{frametitle}{}{%
\begin{tikzpicture}[remember picture,overlay]
\node[anchor=north east,yshift=2pt] at (current page.north east) {\includegraphics[height=1cm]{img/logo_curso}};
\end{tikzpicture}}

\setbeamertemplate{footline}{
	\color{white}
     \\
    }

\begin{frame}
	\maketitle
\end{frame}

\addtocounter{framenumber}{-1}

% Descomente as linhas abaixo se desejar colocar um sumário de todas as seções
%\begin{frame}[t]{Sumário}
%\tableofcontents
%\end{frame}
\setbeamertemplate{footline}{
    \color{black}
    \hfill \insertframenumber\,/\,\inserttotalframenumber~
}

%Utilizado para alinhar figuras
%\setbeamertemplate{caption}{\raggedright\insertcaption\par}
%\setbeamertemplate{caption}{\insertcaption}

\def\sectionname{}
\def\insertsectionnumber{}
\def\subsectionname{}
\def\insertsubsectionnumber{}

\AtBeginSection{\frame{\sectionpage}\addtocounter{framenumber}{-1}}


\AtBeginSubsection{\frame{\subsectionpage}\addtocounter{framenumber}{-1} }
\AtBeginSubsubsection{\frame{\subsubsectionpage}\addtocounter{framenumber}{-1} }


%%%%%%%%%%%%%%%%%%%%%%%%%%%%%%%%%%%%%%%%%%%%
% Inicio do documento
%%%%%%%%%%%%%%%%%%%%%%%%%%%%%%%%%%%%%%%%%%%%

% comando para correção ortográfica
%$ aspell -t -c introducao.tex --encoding=utf-8 --lang=pt_BR

\chapter{Introdução}

\section{Problemática}
O consumo de jogos é crescente nos últimos 5 anos, como podemos ver na figura
\ref{fig:esa_graph_2017} das pesquisas da ESA,
segundo os mesmos, em 2016 o consumidor norte americano gastou em torno de $30.4$
bilhões de dólares na industria dos jogos. No mesmo ano, as companhias de jogos
norte americanas adicionaram mais de 11.7 bilhões de dólares na \textit{GDP}
do país \cite{entertainment2017essential}.
\begin{figure}[H]
%    \captionsetup{justification=raggedright, singlelinecheck=false}
    \centering
    \includegraphics[width=0.5\textwidth]{figuras/ESAGraph2017.png}
    \caption{Consumo com conteúdo de jogos\cite{entertainment2017essential}}
    \label{fig:esa_graph_2017}
\end{figure}

Em contrapartida, os investimentos nos jogos também são crescentes neste
período \cite{entertainment2017essential}, um dos jogos com mais investimentos é
o \textit{GTAV} que em desenvolvimento e marketing gastou cerca de $265$
milhões de dólares \cite{villapaz2013gta}. Os jogos tendem a ser cada vez mais,
detalhados e complexos, desta maneira o custo de desenvolvimento também aumenta.
No desenvolvimento de jogos temos vários profissionais envolvidos para criar
o conteúdo dos jogos, com equipes de programadores, designers,  roteiristas,
entre outros. A força de trabalho dos mesmos costuma ser a parte mais cara da
criação do jogo.

\section{Apresentação}
Uma maneira de conseguir diminuir os gastos no desenvolvimento de conteúdo, é 
gerando os mesmos proceduralmente, as chamadas técnicas de \textit{PCG}. 
\textit{PCG} é usar algoritmos para gerar o conteúdo \cite{shaker2016procedural}.
Uma aplicação bem comum do \textit{PCG} é a criação de relevos e mapas de altura,
desta maneira não é necessário uma pessoa modelar manualmente a altura do
terreno do cenário.

Com o \textit{PCG} para criar o terreno, é possível criar eles com tamanhos
pseudo-infinitos, hoje temos diversos exemplos de jogos que fazem uso dessa técnica
para criar um cenário pseudo-infinitos, entre eles, \textit{Limit Theory}, nele
são criados pseudo-infinitos sistemas planetários, de forma procedural, e em cada
sistema planetário os planetas e seus relevos também são gerados proceduralmente
\cite{abreu1990toward}.

O algoritmo para criar o terreno deve ser implementado conforme as
características do bioma alvo, como exemplo, a implementação de 
\cite{gabrielle2016canion} e \cite{carli2012canion}, ambos geram relevos de
cânions, como podemos visualizar na figura \ref{fig:carli2012result}.
\begin{figure}[H]
%    \captionsetup{justification=raggedright, singlelinecheck=false}
    \centering
    \includegraphics[width=0.9\textwidth]{figuras/carli2012result.png}
    \caption{Resultado do trabalho de \cite{carli2012canion}}
    \label{fig:carli2012result}
\end{figure}

Na técnica de \cite{patel2010polygonal}, que gera ilha proceduralmente, cada uma podendo ter
múltiplos biomas. Para começar é criado um diagrama de voronoi, onde cada \textit{site}
vai representar uma região do mapa, os \textit{sites} são gerados em posições
aleatórias, seguido por Lloyd's algorithm no diagrama para deixar as regiões mais relaxadas \ref{fig:voronoi-2-lloyd}.
Então é usados uma sequencia de ruídos bidimensionais sobre o mapa para decidir
áreas de oceano ou ilha, altura da região e bioma da região \ref{fig:patel2010tablebiomes} e \ref{fig:patel2010biomes}, seguido por ruídos
unidimensionais para as bordas da ilha e fronteiras das regiões ter uma aparência mais natural, 
por fim tendo como resultado final ilustrado na figura. 
\begin{figure}[H]
%    \captionsetup{justification=raggedright, singlelinecheck=false}
    \centering
    \includegraphics[width=0.6\textwidth]{figuras/voronoi-2-lloyd.png}
    \caption{Diagrama de voronoi com Lloyd's aplicado}
    \label{fig:voronoi-2-lloyd}
\end{figure}

\begin{figure}[H]
%    \captionsetup{justification=raggedright, singlelinecheck=false}
    \centering
    \includegraphics[width=0.4\textwidth]{figuras/patel2010tablebiomes.png}
    \caption{Tabela de biomas}
    \label{fig:patel2010tablebiomes}
\end{figure}

\begin{figure}[H]
%    \captionsetup{justification=raggedright, singlelinecheck=false}
    \centering
    \includegraphics[width=0.4\textwidth]{figuras/patel2010biomes.png}
    \caption{Regiões com bioma}
    \label{fig:patel2010biomes}
\end{figure}

\begin{figure}[H]
%    \captionsetup{justification=raggedright, singlelinecheck=false}
    \centering
    \includegraphics[width=0.4\textwidth]{figuras/voronoi-map-goal-distorted.png}
    \caption{Resultado final de \cite{patel2010polygonal}}
    \label{fig:voronoi-map-goal-distorted}
\end{figure}

O jogo \textit{minecraft} tem uma implementação de geração procedural, seus mundos são de tamanho pseudo-infinito, 
o algoritmo é não assistido, ou seja, sem a intervenção do usuário na criação do mundo, nele
são gerados múltiplos biomas, em cada um deles existe uma manipulação diferente
do ruído para recriar as características do bioma. Mesmo em seu mundo
minimalista, o jogador consegue reconhecer o bioma \cite{short2012teaching}, como na figura \ref{fig:biomesminecraftgameplay}.
Foi usado o sistema online \textit{Chunck Base} para 
ilustrar fronteiras entre biomas, cada cor no mapa é um bioma \ref{fig:chunkbasebiomes}.


\begin{figure}[H]
%    \captionsetup{justification=raggedright, singlelinecheck=false}
    \centering
    \includegraphics[width=0.7\textwidth]{figuras/chunkbasebiomes.png}
    \caption{Mapa de biomas para mundo virtual do \textit{minecraft}}
    \label{fig:chunkbasebiomes}
\end{figure}

\begin{figure}[H]
%    \captionsetup{justification=raggedright, singlelinecheck=false}
    \centering
    \includegraphics[width=0.7\textwidth]{figuras/biomesminecraftgameplay.png}
    \caption{Perspectiva de um jogador no \textit{minecraft}}
    \label{fig:biomesminecraftgameplay}
\end{figure}
%\subsubsection{Exemplo de ilustração}

%\begin{figure}[h]
%\captionsetup{justification=raggedright, singlelinecheck=false}
%\centering
%\includegraphics[scale=0.4]{figuras/uffs.eps}
%\caption{Exemplo de ilustração.}
%\label{fig:uffs}
%\end{figure}



%Exemplo do Matheus
%\begin{figure}[H]
%  \centering
%  \includegraphics[width=0.9\textwidth]{figuras/tcc2/pts_sem_peso/evento_de_site_insercao_fronteiras.png}
%  \caption{ES para o \textit{site} $s$.}
%  \label{fig:cenario_insercao_de_fronteiras}
%\end{figure}
%
%  Levando em consideração o cenário da figura \ref{fig:cenario_insercao_de_fronteiras}, 
%$s$ está contido em $R^{*}_{q}$, portanto serão criadas as fronteiras $C^{-}_{qs}$ e $C^{+}_{qs}$,

\begin{frame}{Biomas}
    \begin{itemize} \setlength\itemsep{1em}
        \item Conceito de biomas usado por \cite{walter1986vegetaccao};
        \item O conceito de biomas utilizado neste trabalho.
    \end{itemize}
    \begin{figure}
        \centering
        \begin{subfigure}[b]{0.47\textwidth}
            \includegraphics[width=\textwidth]{img/mineExtremeHills}
            \caption{Bioma \textit{Extreme Hills} no minecraft}
            \label{fig:mineExtremeHills}
        \end{subfigure}
        ~ %add desired spacing between images, e. g. ~, \quad, \qquad, \hfill etc. 
          %(or a blank line to force the subfigure onto a new line)
        \begin{subfigure}[b]{0.47\textwidth}
            \includegraphics[width=\textwidth]{img/minePlains}
            \caption{Bioma \textit{Plains} no minecraft}
            \label{fig:minePlains}
        \end{subfigure}
        ~ %add desired spacing between images, e. g. ~, \quad, \qquad, \hfill etc. 
        %(or a blank line to force the subfigure onto a new line)
        \caption{Exemplo de Biomas no minecraft}
        \label{fig:mineBiomes}
    \end{figure}
    
    
\end{frame}
\begin{frame}{Representações de Regiões}
    \begin{itemize} \setlength\itemsep{1em}
        \item Malha de quadrados
        \begin{figure}[H]
            \centering
            \includegraphics[width=.5\textwidth, height=.5\textheight]{img/squadStripBiomes}
            \caption{Malha de quadrados com divisão entre três Biomas.}
            \label{fig:squadStripBiomes}
        \end{figure}
    \end{itemize}
\end{frame}

\begin{frame}{Representações de Regiões}
    \begin{itemize} \setlength\itemsep{1em}
        \item Malha de triangulos
        
    \end{itemize}
\end{frame}

\begin{frame}{Representações de Regiões}
    \begin{itemize} \setlength\itemsep{1em}
        \item Diagrama de Voronoi
        
    \end{itemize}
\end{frame}
%pdflatex apresentacao.tex 
%bibtex apresentacao.aux
%pdflatex apresentacao.tex 
%pdflatex apresentacao.tex 

\begin{frame}{Ruído de Perlin}
    \begin{itemize}\setlength\itemsep{1em}
        \item Para obter uma forma mais orgânica de relevos, podemos usar a função de ruído.  
    \end{itemize}
    \begin{figure}[H]
        \centering
        \includegraphics[width=.75\textwidth]{img/randomAndNoise}
        \caption{Da esquerda: Função de ruído e função que gera números aleatórios. Por \cite{shiffman2012nature}}
        \label{fig:randomAndNoise}
    \end{figure}
    
\end{frame}

\begin{frame}{Ruído de Perlin}
    \begin{figure}[H]
        \centering
        \includegraphics[width=.75\textwidth]{img/1dto2dnoise}
        \caption{À esquerda relação entre pontos unidimensionais, à direita bidimensionais. Por \cite{shiffman2012nature}}
        \label{fig:1dto2dnoise}
    \end{figure}
    
\end{frame}



\begin{frame}{Ruído de Perlin}
    
    \begin{figure}
        \centering
        \begin{subfigure}[b]{0.6\textwidth}
            \includegraphics[width=\textwidth]{img/perlin1d}
            \caption{Ruído de Perlin com uma dimensão}
            \label{fig:perlin1d}
        \end{subfigure}
        ~ %add desired spacing between images, e. g. ~, \quad, \qquad, \hfill etc. 
          %(or a blank line to force the subfigure onto a new line)
        \begin{subfigure}[b]{0.35\textwidth}
            \includegraphics[width=\textwidth]{img/perlin2d}
            \caption{Ruído de Perlin com duas dimensões}
            \label{fig:perlin2d}
        \end{subfigure}
        ~ %add desired spacing between images, e. g. ~, \quad, \qquad, \hfill etc. 
        %(or a blank line to force the subfigure onto a new line)
        \caption{Perlin para uma e duas dimensões. Por \cite{elias2000perlin}}
        \label{fig:perlin1d2d}
    \end{figure}
    
\end{frame}

\begin{frame}{Ruído de Perlin}
    \begin{itemize}\setlength\itemsep{1em}
        \item Manipulação do ruído de Perlin.  
    \end{itemize}
    \begin{figure}[H]
        \centering
        \includegraphics[width=.75\textwidth]{img/carliResult}
        \caption{Resultado final de \cite{carli2012canion}}
        \label{fig:carliResult}
    \end{figure}
    
\end{frame}
\begin{frame}{Mapas de Altura}
    
\end{frame}
\begin{frame}{Metodologia}
    \begin{itemize}
        \item Implementar malhas da superfície com tamanho pseudo-infinito;
        \item Selecionar biomas, e as características dos mesmos a ser representadas;
        \item Construir algoritmo para manipular ruído de Perlin e gerar características
            selecionadas do bioma;
        \item Gerar divisões entre biomas sobre a malha de regiões;
        \item Implementar fronteiras contínuas entre biomas;
        \item Comparar resultado com cenários de jogos.
    \end{itemize}
\end{frame}


\setbeamertemplate{footline}{
	\color{white}
     \\
    }

\begin{frame}[allowframebreaks]{Referências} % allowframebreaks -> Referências grandes
   \bibliography{referencias}
   \bibliographystyle{apalike}
\end{frame}

\begin{frame}
	\maketitle
\end{frame}

%% Coloquei o  livro em  anotacoes.bib. Pode incluir  outras referências
%% alí. (AG)

\end{document}
